\documentclass[12pt,oneside]{amsart}

\title{MATH 536 Project 4}
\author{James Rushing}
\date{04/22/20}

% Homework template by Zach Teitler
% Public domain, free to use and modify for any purpose
% v1.0 2020-01-13

% Packages

\usepackage[T1]{fontenc}
\usepackage{amsmath,amsfonts,amssymb,amsthm}
\usepackage[letterpaper,margin=1.5in]{geometry}
\usepackage[pagebackref]{hyperref}
\usepackage{booktabs}
\usepackage{enumitem}
\usepackage{pgfplots}
\usepgfplotslibrary{polar}
\usepgflibrary{shapes.geometric}
\usetikzlibrary{calc}
\usepackage{calrsfs}

\usepackage{fancyhdr}
\pagestyle{fancy}



% Extra space between lines

\linespread{2.0}



% Theorems, lemmas, etc.


\newtheorem{theorem}[equation]{Theorem}
\newtheorem{claim}[equation]{Claim}
\newtheorem{lemma}[equation]{Theorem}
\newtheorem{corollary}[equation]{Theorem}
\newtheorem{conjecture}[equation]{Conjecture}
\newtheorem{question}[equation]{Question}

\theoremstyle{definition}

\newtheorem{definition}[equation]{Definition}

\theoremstyle{remark}

\newtheorem{exer}{Exercise}
\newtheorem{remark}[equation]{Remark}
\newtheorem{example}[equation]{Example}

\numberwithin{equation}{exer}


% Answers

% Use "proof" for proof exercises!
% Use "answer" for question (non-proof) exercises

\newenvironment{answer}{\bigskip\noindent\emph{Answer.}}{\hfill$\diamond$\newline}


% Symbols
\DeclareMathAlphabet{\pazocal}{OMS}{zplm}{m}{n}
\newcommand{\bbC}{\mathbb{C}}
\newcommand{\bbN}{\mathbb{N}}
\newcommand{\bbR}{\mathbb{R}}
\newcommand{\bbQ}{\mathbb{Q}}
\newcommand{\bbZ}{\mathbb{Z}}
\newcommand{\Lb}{\pazocal{L}}



\begin{document}
\maketitle
\begin{exer} 
\newline
\begin{answer}

\indent\newline
Given a linear, homogeneous equation with zero flux at the boundaries we can expect a general solution of the following form:
\begin{align}
    v(x,t) = \frac{1}{2} a_0 + \sum_{n=1}^\infty a_n \cos{\frac{n \pi x}{L}}e^{-\varepsilon(n \pi/L)^2t}.
\end{align}

\end{answer}
\end{exer}











\newpage
\begin{exer}

\end{exer}
\begin{answer}
We can expect the particular the solution to have the following form:
\begin{align}
    w(x,t) = (At+B)\cos{\frac{2 \pi x}{L}}+ (Bt +C)\sin{\frac{ 2 \pi x}{L}}
\end{align}
With 2.1, we can calculate that pertinent partial derivatives:
\begin{align}
    \frac{\partial w}{\partial t} = A\cos{\frac{2 \pi x}{L}} + C\sin{\frac{2 \pi x}{L}}
\end{align}
\begin{align}
    \frac{\partial w}{\partial x} = -\frac{2 \pi}{L}\Big(At+B\Big)\sin{\frac{2 \pi x}{L}} +\frac{2 \pi}{L}\Big(Ct+D\Big)\cos{\frac{2 \pi x}{L}}
\end{align}
\begin{align}
    \frac{\partial^2 w}{\partial x^2} = -\frac{4 \pi^2}{L^2}\Big(At+B\Big)\cos{\frac{2 \pi x}{L}} - \frac{4 \pi^2}{L^2} \Big(Ct+D\Big)\sin{\frac{2 \pi x}{L}}.
\end{align}
Now we plug 2.4 and 2.2 into 2.1 and compare coefficients. 
\begin{align*}
    A\cos{\frac{2 \pi x}{L}} + C\sin{\frac{2 \pi x}{L}} = -\frac{4\varepsilon \pi^2}{L^2}\Big[(At+B)\cos{\frac{2 \pi x}{L}}+ (Bt +C)\sin{\frac{ 2 \pi x}{L}}\Big] + t\cos{\frac{2 \pi x}{L}}\\
    \implies -\frac{AL^2}{4\varepsilon\pi^2}\cos{\frac{2 \pi x}{L}} - -\frac{CL^2}{4\varepsilon\pi^2}\sin{\frac{2 \pi x}{L}}+\frac{tL^2}{4\varepsilon\pi^2}\cos{\frac{2 \pi x}{L}}= (At+B)\cos{\frac{2 \pi x}{L}}+ (Bt +C)\sin{\frac{ 2 \pi x}{L}}\\
    \implies \Bigg( -\frac{AL^2}{4\varepsilon\pi^2} - At -B + \frac{tL^2}{4\varepsilon\pi^2}\Bigg)\cos{\frac{2 \pi x}{L}} + \Bigg(-\frac{CL^2}{4\varepsilon\pi^2}-Ct-D\Bigg)\sin{\frac{2 \pi x}{L}}=0.
\end{align*}
Now, both coefficients must be zero. This mean the linear and constant terms of both coefficients must be zero. We see expressly that C and D are zero. For A and B we observe the following:
\begin{align}
    \frac{L^2}{4\varepsilon\pi^2} - A = 0 \implies A = \frac{L^2}{4\varepsilon\pi^2} 
    \implies B= -\frac{AL^2}{4\varepsilon\pi^2} \implies B = -\frac{L^4}{16\varepsilon^2\pi^4}.
\end{align}
Combining 2.5 and 2.1 we have a particular as follows:
\begin{align*}
      w(x,t) = \Big(\frac{L^2}{4\varepsilon\pi^2}t-\frac{L^4}{16\varepsilon^2\pi^4}\Big)\cos{\frac{2 \pi x}{L}}
\end{align*}
\end{answer}

\begin{exer}
\begin{answer}
Combining $v$ and $w$ as $u$ we get the following class of solutions:
\begin{align*}
    u(x,t) = \frac{1}{2} a_0 + \sum_{n=1}^\infty a_n \cos{\frac{n \pi x}{L}}e^{-\varepsilon(n \pi/L)^2t} + \Big(\frac{L^2}{4\varepsilon\pi^2}t-\frac{L^4}{16\varepsilon^2\pi^4}\Big)\cos{\frac{2 \pi x}{L}}
\end{align*}
Computing the partial of the above with respect to $x$ allows us to check the boundary condidtions. 
\begin{align}
\frac{\partial u}{\partial x}= -\frac{n\pi}{L} \sum_{n=1}^\infty a_n \sin{\frac{n \pi x}{L}}e^{-\varepsilon(n \pi/L)^2t} - \frac{2\pi}{L}\Big(\frac{L^2}{4\varepsilon\pi^2}t-\frac{L^4}{16\varepsilon^2\pi^4}\Big)\sin{\frac{2 \pi x}{L}}.
\end{align}
We can see that 3.1 satisfies the zero flux condition because we have a combination of $\sin$ functions that evaluate to zero when $x=L=0$. Finally, using the initial condition we see the following:
\begin{align*}
    u(x,0)= \cos{\frac{2 \pi x}{L}} = \frac{1}{2}  a_0 +\sum_{n=1}^\infty a_n \cos{\frac{2 \pi x}{L}} -\frac{L^2}{4\varepsilon\pi^2} \cos{\frac{2 \pi x}{L}}
    \implies a_2 = 1+ \frac{L^2}{4\varepsilon\pi^2}.
\end{align*}
Furthermore, we see all other coefficients are zero. Therefore, our unique solution is as follows:
\begin{align*}
    c(x,t) = \Bigg( 1+ \frac{L^2}{4\varepsilon\pi^2}\Bigg) \cos{\frac{2 \pi x}{L}}e^{-\varepsilon(n \pi/L)^2t} + \Big(\frac{L^2}{4\varepsilon\pi^2}t-\frac{L^4}{16\varepsilon^2\pi^4}\Big)\cos{\frac{2 \pi x}{L}}.
\end{align*}
\end{answer}
\end{exer}
\end{document}
 