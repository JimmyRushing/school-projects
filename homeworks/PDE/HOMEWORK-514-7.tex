\documentclass[12pt,oneside]{amsart}

\title{Math 414/514 Homework 7}
\author{James Rushing}
\date{04/12/20}

% Homework template by Zach Teitler
% Public domain, free to use and modify for any purpose
% v1.0 2020-01-13

% Packages

\usepackage[T1]{fontenc}
\usepackage{amsmath,amsfonts,amssymb,amsthm}
\usepackage[letterpaper,margin=1.5in]{geometry}
\usepackage[pagebackref]{hyperref}
\usepackage{booktabs}
\usepackage{enumitem}

\usepackage{fancyhdr}
\pagestyle{fancy}



% Extra space between lines

\linespread{1.6}



% Theorems, lemmas, etc.


\newtheorem{theorem}[equation]{Theorem}
\newtheorem{claim}[equation]{Claim}
\newtheorem{lemma}[equation]{Theorem}
\newtheorem{corollary}[equation]{Theorem}
\newtheorem{conjecture}[equation]{Conjecture}
\newtheorem{question}[equation]{Question}

\theoremstyle{definition}

\newtheorem{definition}[equation]{Definition}

\theoremstyle{remark}

\newtheorem{exer}{Exercise}
\newtheorem{remark}[equation]{Remark}
\newtheorem{example}[equation]{Example}

\numberwithin{equation}{exer}


% Answers

% Use "proof" for proof exercises!
% Use "answer" for question (non-proof) exercises

\newenvironment{answer}{\bigskip\noindent\emph{Answer.}}{\hfill$\diamond$\newline}


% Symbols

\newcommand{\bbC}{\mathbb{C}}
\newcommand{\bbN}{\mathbb{N}}
\newcommand{\bbR}{\mathbb{R}}
\newcommand{\bbQ}{\mathbb{Q}}
\newcommand{\bbZ}{\mathbb{Z}}
\newcommand{\smm}{\sum_{k=1}^n}
\newcommand{\snn}{\sin{\big(\frac{t}{2}\big)}}


\begin{document}
\maketitle

\begin{exer} (10.8.6)
\newline \indent 
Establish the following properties of the Dirichlet kernel, $D_n(t)$.
\end{exer}
\begin{answer}

\begin{enumerate}[label=(\alph*)]
\item $D_n(t)$ is a sum of continuous function and is therefore continuous (5.4). The following shows that $D_n(t)$ is $2 \pi$ periodic. 
\begin{align*}
    D_n(t+2 \pi)= \frac{1}{2} + \smm\cos{(t(t+2 \pi))}= \frac{1}{2} + \smm \cos{(kt+2k \pi)}\\
    = \frac{1}{2} + \smm \big[\cos{(kt)}\cos{(2k\pi)} - \sin{(kt)}\sin{(2k\pi)}\big] = \frac{1}{2} + \smm\cos{(kt)} +0\\
    =\frac{1}{2} + \smm\cos{(kt)}=D_n(t).
\end{align*}
\item $D_n(t)$ is an even function.
\begin{align*}
    D_n(-t)= \frac{1}{2} + \smm \cos{(-kt)} = \frac{1}{2} + \smm \cos{(kt)}= D_n(t).
\end{align*}
\item Because of (b) we can calculate the following integral.
\begin{align*}
    \frac{1}{\pi} \int_\pi^\pi D_n(t)dt = \frac{2}{\pi}\int_0^\pi D_n(t)dt
    = \frac{2}{\pi} \bigg[\int_0^\pi \frac{1}{2}dt + \int_0^\pi \smm \cos{(kt)}dt\bigg]\\
    = \frac{2}{\pi}\Bigg[\frac{1}{2} [t]_0^\pi + \bigg[\frac{\smm \sin{(kt)}}{k}\bigg]_0^\pi \Bigg]= \frac{2}{\pi}\bigg[\frac{1}{2} [\pi] +0 \bigg]=1.
\end{align*}

\item To reveal an alternate form of $D_n(t)$ we multiply through by $2\snn$ and apply trig identities to simplify by recognising a telescoping sum.
\begin{align*}
    D_n(t) = \frac{1}{2} + \smm \cos{(kt)}\\
    \implies 2\sin{\big(\frac{t}{2}\big)} D_n(t)
    = \sin{\big(\frac{t}{2}\big)} + \smm2\sin{(\frac{1}{2})}\cos{(kt)}\\
    = \sin{\big(\frac{t}{2}\big)} + \smm\bigg[ \sin{(\frac{t}{2} + kt)} + \sin{(\frac{t}{2} - kt)}\bigg]\\
    = \snn + \sin{(\frac{3t}{2})} - \snn +\sin{(\frac{5t}{2})} - \sin{(\frac{3t}{2})} +\dots+
    =\sin{(t(n+\frac{1}{2}))}\\
    \implies D_n(t) = \frac{\sin{(t(n+\frac{1}{2}))}}{2\snn}.
\end{align*}
\item Knowing that $|\cos{x}|$ \leq 1 \indent $\forall \indent x \in \mathbb{R}$
\begin{align*}
    D_n(0) = \frac{1}{2} + \smm \cos{(0)} = \frac{1}{2} + 1 * n =\frac{1}{2} +n 
\end{align*}
\item
\begin{align*}
|D_n(t)| =\big| \frac{1}{2} + \smm \cos{(kt)} \big| \leq \frac{1}{2} + 1*n = n +\frac{1}{2}
\end{align*}
\item Knowing that $|\sin{x}|$ \leq 1 \indent $\forall \indent x \in \mathbb{R}$, we have that
\begin{align*}
    |D_n(t)| = \Bigg| \frac{\sin{(t(n+\frac{1}{2}))}}{2\snn} \Bigg| \leq \frac{1}{2}.
\end{align*}
Where $0 < |t| < \pi$ we have $0<\frac{1}{2} < \frac{\pi}{2|t|}$ which implies that
\begin{align*}
    |D_n(t)| \leq \frac{\pi}{2|t|}.
\end{align*}

\end{enumerate}

\end{answer}




\newpage
\begin{exer}(10.8.11 + 10.8.14)
\indent Knowing that for any continuous function, $f$, we have a sequence of polynomials converging to $f$ doesn't tell us that there is also a power series that converges to $f$. A power series must be expressible in terms of one sequence of coefficients. In other words, one infinite polynomial won't do the job of a sequence of polynomials. 
\newline \indent 
We can not apply the WAT to functions that are continuous on an unbounded interval. It works on compact intervals because a function that is continuous on a compact interval is uniformly continuous there. $f(x) = \frac{1}{x}$ is continuous on $(0,1)$ but is not well behaved enough near $0$ to be approximated using polynomials. 
\end{exer}







\newpage \indent
\begin{exer}(10.8.15)

Let $\int_0^1 f(x) x^n dx=0$ $\forall$ $n \in \mathbb{N}$ where $f:[0,1] \to \mathbb{R}$ is a continuous function. Prove that $f(x) \equiv 0$ $\forall x \in [0,1]$.
\end{exer}
\begin{proof}
If $\int_0^1 f(x) x^n dx=0$ $\forall$ $n \in \mathbb{N}$ then by the linear property of integrals(8.5) we must have that 
\begin{align*}
   \int_0^1 f(x) g_n(x) dx=0
\end{align*}
$\forall$ $n \in \mathbb{N}$ and where $g_n(x)$ represents the sequence of polynomials of whose existence we are assured of by the WAT (10.37). Knowing this, we sneak this integral integral below as follows:
\begin{align*}
    \Big|\int_0^1 (f(x))^2\Big|=\Big|\int_0^1 (f(x))^2-\int_0^1 f(x) g_n(x) dx\Big|= 
    \Big| \int_0^1 (f(x))^2- f(x) g_n(x) dx\Big|.
\end{align*}
The absolute value property of integrals (8.7) followed by a basic property of absolute values (1.17) gives us the following inequality:
\begin{align*}
    \Big| \int_0^1 (f(x))^2- f(x) g_n(x) dx\Big| \leq
    \int_0^1 \Big| (f(x))^2- f(x) g_n(x) dx \Big|\\
    = \int_0^1 \Big|f(x) \Big( f(x) -g_n(x)\Big)\Big| dx =
    \int_0^1 \Big|f(x)\Big|\Big|f(x)-g_n(x)\Big|dx
\end{align*}
Now, because $f(x)$ is continuous on the bounded interval $[0,1]$, we have that there exists an $M$ such that $\Big|f(x)\Big| \leq M$ $\forall$ $x \in [0,1]$. Furthermore, by the WAT, and letting $\frac{\epsilon}{M}>0$, combining all of the above results, we have the following:
\begin{align*}
        \Big|\int_0^1 (f(x))^2\Big| \leq
        \int_0^1 \Big|f(x)\Big|\Big|f(x)-g_n(x)\Big|dx \leq
        1*M*\frac{\epsilon}{M} = \epsilon\\
        \implies \Big|\int_0^1 (f(x))^2\Big| = 0\\
        \implies f(x) \equiv 0 .
\end{align*}
\end{proof}
\end{document}
