\documentclass[12pt,oneside]{amsart}

\title{Math 414/514 Homework 0}
\author{James Rushing}
\date{01/17/20}

% Homework template by Zach Teitler
% Public domain, free to use and modify for any purpose
% v1.0 2020-01-13

% Packages

\usepackage[T1]{fontenc}
\usepackage{amsmath,amsfonts,amssymb,amsthm}
\usepackage[letterpaper,margin=1.5in]{geometry}
\usepackage[pagebackref]{hyperref}
\usepackage{booktabs}
\usepackage{enumitem}

\usepackage{fancyhdr}
\pagestyle{fancy}



% Extra space between lines

\linespread{1.2}



% Theorems, lemmas, etc.


\newtheorem{theorem}[equation]{Theorem}
\newtheorem{claim}[equation]{Claim}
\newtheorem{lemma}[equation]{Theorem}
\newtheorem{corollary}[equation]{Theorem}
\newtheorem{conjecture}[equation]{Conjecture}
\newtheorem{question}[equation]{Question}

\theoremstyle{definition}

\newtheorem{definition}[equation]{Definition}

\theoremstyle{remark}

\newtheorem{exer}{Exercise}
\newtheorem{remark}[equation]{Remark}
\newtheorem{example}[equation]{Example}

\numberwithin{equation}{exer}


% Answers

% Use "proof" for proof exercises!
% Use "answer" for question (non-proof) exercises

\newenvironment{answer}{\bigskip\noindent\emph{Answer.}}{\hfill$\diamond$\newline}


% Symbols

\newcommand{\bbC}{\mathbb{C}}
\newcommand{\bbN}{\mathbb{N}}
\newcommand{\bbR}{\mathbb{R}}
\newcommand{\bbQ}{\mathbb{Q}}
\newcommand{\bbZ}{\mathbb{Z}}




\begin{document}
\maketitle

\begin{exer}

What is the definition of $\lim_{x \to c} f(x) = L$, using $\epsilon$ and $\delta$?

\end{exer}



\




\begin{answer}

Let $f(x) : E \to \mathbb{R} $ be a function with domain $E$ and suppose that $c$ is a point of accumulation of $E$. Then we write

\begin{align*}
    \lim_{x \to c} f(x) = L
\end{align*}

if $\forall \:  \epsilon > 0 \: \exists\: \delta > 0\: s.t.$

\begin{align*}
    |f(x) - L| < \epsilon
\end{align*}

\

when $x$ is a point of $E$ that is not c and $|x-c| <\delta$.

\end{answer}


\newpage

\begin{exer}

The derivative of $x^2$ is $2x$.

\end{exer}








\begin{answer}
    Gibbs energy is defined as
    \begin{equation}
        G = H - TS. \label{eq:4.14.G}
    \end{equation}

    Gibbs energy of a gas is a function of temperature and pressure.  More
    precisely, $H$ can be expressed with $H(T)$ and $S$ with $S(T, P)$ given
    its standard entropy $S_{298}^\circ$.

    For a diatomic gas, its enthalpy is given as
    \begin{equation}
        H = C_P T = \frac72 RT. \label{eq:4.14.H}
    \end{equation}

    It is somewhat more complicated to calculate entropy.  Consider the
    differential of entropy:
    \begin{align*}
        dS &= \frac{C_P}{T} dT - V\alpha dP \\
        &= \frac{7R}{2T} dT - \frac{R}{P} dP.
    \end{align*}

    Therefore, we can find entropy with integration.
    \[ S(T, P) - S_{298}^\circ = \int_{\SI{298}{\kelvin}}^T \frac{7R}{2T} dT - \int_{\SI{1}{\atm}}^P \frac{R}{P} dP.\]
    \begin{equation}
        S = S_{298}^\circ + R \left( \frac72 \ln\frac{T}{\SI{298}{\kelvin}} - \ln\frac{P}{\SI{1}{\atm}} \right)
        \label{eq:4.14.S}
    \end{equation}

    Combine equation \eqref{eq:4.14.G}, \eqref{eq:4.14.H}, and
    \eqref{eq:4.14.S}:
    \[ G = \frac72 RT - S_{298}^\circ T + RT \left( \ln\frac{P}{\SI{1}{\atm}} - \frac72 \ln\frac{T}{\SI{298}{\kelvin}} \right) \]

    Plot this function given $S_{298}^\circ = \SI{130.57}{\joule\per\mol\per\kelvin}$
    and $R = \SI{8.314}{\joule\per\mol\per\kelvin}$.  This binary function,
    whose plot is a surface in a 3-D space, is preferably plotted with a
    computer program.
\end{answer}








\begin{proof}

Using the limit definition of derivative, we have
\begin{equation*} \label{eq1}
\begin{split}
 \lim_{h \to 0} \frac{(x+h)^2 - x^2}{h}
    &= \lim_{h \to 0} \frac{x^2 + 2xh + h^2- x^2}{h}\\
    &= \lim_{h \to 0}\frac{2xh + h^2}{h}\\
    &= \lim_{h \to 0}2x + h = 2x
\end{split}
\end{equation*}




  

\end{proof}

\newpage



\begin{exer}

Some students write quotation marks like this:

"quotation marks" or like this: ''quotation marks''.

Why are these bad, and what is the right way to write quotation marks in \LaTeX?

\end{exer}

\begin{answer}

``The proper way to create quotations is by using the key to the left of the number `1' key for the opening quotes, and to use the apostrophe for the closing quotes. '' - James Rushing. \\ \indent The aforementioned methods are bad because \LaTeX \: always treats `"' as closing quotations. 

\end{answer}


\end{document}
