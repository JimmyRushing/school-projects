\documentclass[12pt,oneside]{amsart}

\title{MATH 536 Project 2}
\author{James Rushing}
\date{02/28/20}

% Homework template by Zach Teitler
% Public domain, free to use and modify for any purpose
% v1.0 2020-01-13

% Packages

\usepackage[T1]{fontenc}
\usepackage{amsmath,amsfonts,amssymb,amsthm}
\usepackage[letterpaper,margin=1.5in]{geometry}
\usepackage[pagebackref]{hyperref}
\usepackage{booktabs}
\usepackage{enumitem}
\usepackage{pgfplots}
\usepgfplotslibrary{polar}
\usepgflibrary{shapes.geometric}
\usetikzlibrary{calc}
\usepackage{calrsfs}

\usepackage{fancyhdr}
\pagestyle{fancy}



% Extra space between lines

\linespread{2.0}



% Theorems, lemmas, etc.


\newtheorem{theorem}[equation]{Theorem}
\newtheorem{claim}[equation]{Claim}
\newtheorem{lemma}[equation]{Theorem}
\newtheorem{corollary}[equation]{Theorem}
\newtheorem{conjecture}[equation]{Conjecture}
\newtheorem{question}[equation]{Question}

\theoremstyle{definition}

\newtheorem{definition}[equation]{Definition}

\theoremstyle{remark}

\newtheorem{exer}{Exercise}
\newtheorem{remark}[equation]{Remark}
\newtheorem{example}[equation]{Example}

\numberwithin{equation}{exer}


% Answers

% Use "proof" for proof exercises!
% Use "answer" for question (non-proof) exercises

\newenvironment{answer}{\bigskip\noindent\emph{Answer.}}{\hfill$\diamond$\newline}


% Symbols
\DeclareMathAlphabet{\pazocal}{OMS}{zplm}{m}{n}
\newcommand{\bbC}{\mathbb{C}}
\newcommand{\bbN}{\mathbb{N}}
\newcommand{\bbR}{\mathbb{R}}
\newcommand{\bbQ}{\mathbb{Q}}
\newcommand{\bbZ}{\mathbb{Z}}
\newcommand{\Lb}{\pazocal{L}}



\begin{document}
\maketitle
\begin{exer} 
\indent\newline
Consider the differential operator
\begin{align*}
    \Lb = 2 \frac{\partial^2}{\partial x^2} + \frac{\partial^2}{\partial x \partial t} - \frac{\partial^2}{\partial t^2}.
\end{align*}
First, we rewrite the middle operator as
\begin{align*}
-\frac{\partial}{\partial x \partial t} + 2\frac{\partial}{\partial t \partial x}
\end{align*}
then we observe the following steps:
\begin{align*}
    \Lb = 2 \frac{\partial^2}{\partial x^2}  -\frac{\partial}{\partial x \partial t} + 2\frac{\partial}{\partial t \partial x}- \frac{\partial^2}{\partial t^2}\\
    = \frac{\partial}{\partial x}\Bigg(2\frac{\partial}{\partial x} - \frac{\partial}{\partial t}\Bigg) + \frac{\partial}{\partial x}\Bigg(2\frac{\partial}{\partial x} - \frac{\partial}{\partial t^2}\Bigg)\\
    = \Bigg(\frac{\partial}{\partial x} + \frac{\partial}{\partial t}\Bigg)\Bigg(2\frac{\partial}{\partial x} - \frac{\partial}{\partial t}\Bigg).
\end{align*}
\end{exer}
\newpage
\indent \newline

\begin{exer}
Defining $\xi = -x + t$ and $\eta = \frac{x}{2} + t$ we have
\begin{align*}
    \frac{\partial y}{\partial t} 
    = \frac{\partial \tilde{y}}{\partial t} \Big(\xi , \eta\Big)
    =\frac{\partial \tilde{y}}{\partial t}\Big(-x + t,\frac{x}{2} + t \Big)
    =\frac{\partial \tilde{y}}{\partial \xi} \cdot \frac{\partial \xi}{\partial t} + \frac{\partial \tilde{y}}{\partial \eta} \cdot \frac{\partial \eta}{\partial t}\\
    =\frac{\partial \tilde{y}}{\partial \xi} \cdot (1) + \frac{\partial \tilde{y}}{\partial \eta} \cdot (1)
    = \Big(\frac{\partial}{\partial \xi} +\frac{\partial}{\partial \eta}\Big) \tilde{y}
\end{align*}
\begin{align*}
    \textbf{and}
\end{align*}
\begin{align*}
    \frac{\partial y}{\partial x} 
    = \frac{\partial \tilde{y}}{\partial x} \Big(\xi , \eta\Big)
    =\frac{\partial \tilde{y}}{\partial x}\Big(-x + t,\frac{x}{2} + t \Big)
    =\frac{\partial \tilde{y}}{\partial \xi} \cdot \frac{\partial \xi}{\partial x} + \frac{\partial \tilde{y}}{\partial \eta} \cdot \frac{\partial \eta}{\partial x}\\
    =\frac{\partial \tilde{y}}{\partial \xi} \cdot (-1) + \frac{\partial \tilde{y}}{\partial \eta} \cdot \frac{1}{2} = \Big(-\frac{\partial}{\partial \xi} + \frac{1}{2} \frac{\partial}{\partial \eta}\Big)\tilde{y}.
\end{align*}

\end{exer}
\newpage
\indent \newline
\begin{exer}
Now we translate our operators from \emph{Exercise 1} in terms of $\xi$ and $\eta$ as follows:
\end{exer}
\begin{align*}
    \frac{\partial}{\partial x} + \frac{\partial}{\partial t} = -\frac{\partial}{\partial \xi} + \frac{1}{2}\cdot \frac{\partial}{\partial \eta} + \frac{\partial}{\partial \xi} + \frac{\partial}{\partial \eta}
    = \frac{3}{2}\cdot \frac{\partial}{\partial \eta}
\end{align*}
\begin{align*}
    \textbf{and}
\end{align*}
\begin{align*}
    2\frac{\partial}{\partial x} - \frac{\partial}{\partial t} 
    = 2\Big(-\frac{\partial}{\partial \xi} + \frac{1}{2}\cdot \frac{\partial}{\partial \eta}\Big)-\Big(\frac{\partial}{\partial \xi} + \frac{\partial}{\partial \eta}\Big)\\
    = -2\frac{\partial}{\partial \xi} + \frac{\partial}{\partial \eta} - \frac{\partial}{\partial \xi} - \frac{\partial}{\partial \eta}
    =-3\frac{\partial}{\partial \xi}.
\end{align*}
\newpage
\indent \newline
\begin{exer}
Writing our differential operator in terms of $\xi$ and $\eta$ gives
\end{exer}
\begin{align*}
    \Lb
    = \Bigg(\frac{\partial}{\partial x} + \frac{\partial}{\partial t}\Bigg)\Bigg(2\frac{\partial}{\partial x} - \frac{\partial}{\partial t}\Bigg) =  \Big(\frac{3}{2}\cdot \frac{\partial}{\partial \eta}\Big)\Big(-3\frac{\partial}{\partial \xi}\Big) = -\frac{9}{2}\frac{\partial}{\partial \eta}\frac{\partial}{\partial \xi}.
\end{align*}
With this, we have 
\begin{align*}
        2 \frac{\partial^2 y}{\partial x^2} + \frac{\partial^2 y}{\partial x \partial t} - \frac{\partial^2 y}{\partial t^2} =0\\
        \implies  \Big(-\frac{9}{2}\frac{\partial}{\partial \eta}\frac{\partial}{\partial \xi}\Big)\tilde{y} = 0
\end{align*}
This tells us that the partial of $\tilde{y}$ in respect to $\xi$ is independent of $\eta$. Using this information we write it as a function of one variable and, using a dummy variable, we integrate as follows to reveal the the general form of $y(x,t)$:
\begin{align*}
    \frac{\partial \tilde{y}}{\partial \xi} = h(\xi)\\
    \implies \int_0^\xi \frac{\partial \tilde{y}}{\partial \xi'} d\xi'= \int_0^\xi h(\xi')d\xi'\\
    \implies \Big[\tilde{y}(\xi,\eta)\Big]_0^\xi = \int_0^\xi h(\xi')d\xi'\\
    \implies y(x,t) = \tilde{y}(\xi,\eta) = \tilde{y}(0,\eta) + \int_0^\xi h(\xi')d\xi' \\
    =f(\eta) + g(\xi) = f\big(\frac{x}{2} + t \big) + g (-x+t).
\end{align*}
\newpage
\begin{exer}
Take initial conditions to be the following:
\begin{align*}
    y(x,0)= \sin{x}\\
    y_t(x,0) = 0.
\end{align*}
\end{exer}
\begin{answer}
Expressly from our derivation of $y(x,t)$ we have that 
\begin{align*}
    y(x,0) = f(\frac{x}{2}) + g(-x) = 0.
\end{align*}
Now, to satisfying our second initial condition we employ the FTC to reveal a second linear combination of $f$ and $g$. 
\begin{align*}
    y_t(x,0) = 0 = f'(\frac{x}{2} + 0) + g'(-x + 0) = f'(\frac{x}{2}) + g'(-x)\\
    \implies \int 0 = \int f'(\frac{x}{2}) + \int g'(-x)\\
    \implies C = 2f(\frac{x}{2}) - g(-x)
\end{align*}
for some constant $C$. Combing our two linear combinations of $f$ and $g$ gives us 
\begin{align*}
    f(\frac{x}{2}) =\frac{1}{3}(C + \sin{x}) \indent \textbf{and} \indent
    g(x) = \sin{x} -\frac{2}{3}(C + \sin{x})\\
    \implies y(x,t) = \frac{1}{3}(C + \sin{(\frac{x}{2} + t})) + \sin{(-x + t)} -\frac{2}{3}(C + \sin{(-x + t)}).
\end{align*}
\end{answer}

\end{document}
 