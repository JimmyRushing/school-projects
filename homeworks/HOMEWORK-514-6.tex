\documentclass[12pt,oneside]{amsart}

\title{Math 414/514 Homework 6}
\author{James Rushing}
\date{03/20/20}

% Homework template by Zach Teitler
% Public domain, free to use and modify for any purpose
% v1.0 2020-01-13

% Packages

\usepackage[T1]{fontenc}
\usepackage{amsmath,amsfonts,amssymb,amsthm}
\usepackage[letterpaper,margin=1.5in]{geometry}
\usepackage[pagebackref]{hyperref}
\usepackage{booktabs}
\usepackage{enumitem}

\usepackage{fancyhdr}
\pagestyle{fancy}



% Extra space between lines

\linespread{1.6}



% Theorems, lemmas, etc.


\newtheorem{theorem}[equation]{Theorem}
\newtheorem{claim}[equation]{Claim}
\newtheorem{lemma}[equation]{Theorem}
\newtheorem{corollary}[equation]{Theorem}
\newtheorem{conjecture}[equation]{Conjecture}
\newtheorem{question}[equation]{Question}

\theoremstyle{definition}

\newtheorem{definition}[equation]{Definition}

\theoremstyle{remark}

\newtheorem{exer}{Exercise}
\newtheorem{remark}[equation]{Remark}
\newtheorem{example}[equation]{Example}

\numberwithin{equation}{exer}


% Answers

% Use "proof" for proof exercises!
% Use "answer" for question (non-proof) exercises

\newenvironment{answer}{\bigskip\noindent\emph{Answer.}}{\hfill$\diamond$\newline}


% Symbols

\newcommand{\bbC}{\mathbb{C}}
\newcommand{\bbN}{\mathbb{N}}
\newcommand{\bbR}{\mathbb{R}}
\newcommand{\bbQ}{\mathbb{Q}}
\newcommand{\bbZ}{\mathbb{Z}}
\newcommand{\itt}{\int_0^\pi}
\newcommand{\smm}{\sum_{n=1}^{\infty}}
\newcommand{\ite}{\int_0^1}

\begin{document}
\maketitle

\begin{exer} (9.5.2)
Prove that $\int_0^\pi \sum_{n=1}^\infty \frac{\sin{nx}}{n x}dx 
= \sum_{n=1}^\infty \frac{2}{(2n-1)^3}$
\end{exer}


\begin{proof}

\indent As we have seen in example 9.18, because $\sin{(nx)} \leq 1$ we have that 
\begin{align*}
    \bigg|\frac{\sin{nx}}{n^2}\bigg| \leq \frac{1}{n^2} \indent \forall x \in \mathbb{R}.
\end{align*}
Therefore, by the M-test $ \sum_{n=1}^\infty \frac{\sin{nx}}{n x}dx $ converges for all $x \in \mathbb{R}$.

Knowing that the integrand converges we can move the integral through the sum and integrate term-by-term as follows:
\begin{align*}
        \itt \smm \frac{\sin{nx}}{n^2} dx = \smm \itt \frac{\sin nx}{n^2} dx 
        = \smm \frac{1}{n^2} \itt \sin{nx}dx \\
        =\smm -\frac{1}{n^3}\big[ \cos{nx}\big]_0^\pi =
        \smm - \frac{1}{n^3}\big[\cos{n \pi} - \cos{0} \big] \\
        = \smm -\frac{1}{n^3} \big[(-1)^n - 1\big] = \smm \frac{(-1)^{n+1} + 1}{n^3} \\
        = \frac{2}{1^3} + 0 + \frac{2}{3^3} + 0 +\frac{2}{5^3} +0 + \frac{2}{7^3} = \smm \frac{2}{(2n - 1)^3}.
\end{align*}
\end{proof}









\newpage \indent \newline
\begin{exer} (10.4.4)
\indent Show that 
\begin{align*}
    f(x) = \ite \frac{1 - e^{-sx}}{s} ds =
    \sum_{k=1}^\infty \frac{(-1)^{k-1}x^k}{k(k!)} .
\end{align*}
\end{exer}
\begin{answer}
\indent To begin, we will show that the $kth$ derivative of $f(x)$ can be written as 
\begin{align*}
    f^k(x) = \ite (-1)^{k-1} s^{k-1} e^{-sx}ds.
\end{align*}
By induction, where $k=1$ we have
\begin{align*}
    f'(x) = \ite \big(\frac{d}{dx}s^{-1} - \frac{d}{dx} s^{-1} e^{-sx} \big)ds = \ite e^{-sx}ds = \ite (-1)^{1-1} s ^{1-1} e^{-sx}
\end{align*}
and assuming the equality hold for $k$ we have for that for $k+1$
\begin{align*}
    f^{k+1}(x) = \frac{d}{dx} \ite (-1)^{k-1}s^{k-1} e^{-sx}ds = \ite (-s)(-1)^{k-1}s^{k-1} e^{-sx}ds \\
    = \ite (-1)^{k-1 +1}s^{k-1+1} e^{-sx}ds = \ite (-1)^{(k+1) -1}s^{(k+1)-1} e^{-sx}ds.
\end{align*}    
By theorem 10.17 we have that the coefficients of the power series around $x=0$ for $f(x)$ are as follows:
\begin{align*}
    a_k=\frac{f^k(0)}{k!} \indent \textit{where}\\
    f^k(0) = \ite (-1)^{k-1}s^{k-1} e^0 ds = (-1)^{k-1} \bigg[\frac{s^{k-1+1}}{k-1+1}\bigg]_0^1 = \frac{(-1)^{k-1}}{k}\\
    \implies a_k = \frac{(-1)^{k-1}}{k(k!)}.
\end{align*}
Which was to be shown.
\end{answer}
\end{document}
