\documentclass[12pt,oneside]{amsart}

\title{Math 414/514 Homework 3}
\author{James Rushing}
\date{02/07/20}

% Homework template by Zach Teitler
% Public domain, free to use and modify for any purpose
% v1.0 2020-01-13

% Packages

\usepackage[T1]{fontenc}
\usepackage{amsmath,amsfonts,amssymb,amsthm}
\usepackage[letterpaper,margin=1.5in]{geometry}
\usepackage[pagebackref]{hyperref}
\usepackage{booktabs}
\usepackage{enumitem}

\usepackage{fancyhdr}
\pagestyle{fancy}



% Extra space between lines

\linespread{1.2}



% Theorems, lemmas, etc.


\newtheorem{theorem}[equation]{Theorem}
\newtheorem{claim}[equation]{Claim}
\newtheorem{lemma}[equation]{Theorem}
\newtheorem{corollary}[equation]{Theorem}
\newtheorem{conjecture}[equation]{Conjecture}
\newtheorem{question}[equation]{Question}

\theoremstyle{definition}

\newtheorem{definition}[equation]{Definition}

\theoremstyle{remark}

\newtheorem{exer}{Exercise}
\newtheorem{remark}[equation]{Remark}
\newtheorem{example}[equation]{Example}

\numberwithin{equation}{exer}


% Answers

% Use "proof" for proof exercises!
% Use "answer" for question (non-proof) exercises

\newenvironment{answer}{\bigskip\noindent\emph{Answer.}}{\hfill$\diamond$\newline}


% Symbols

\newcommand{\bbC}{\mathbb{C}}
\newcommand{\bbN}{\mathbb{N}}
\newcommand{\bbR}{\mathbb{R}}
\newcommand{\bbQ}{\mathbb{Q}}
\newcommand{\bbZ}{\mathbb{Z}}




\begin{document}
\maketitle

\begin{exer}

For what values of $p$ does $\int_1^{\infty}x^{-p}dx$ converge?
\end{exer}



\




\begin{answer}

To avoid dividing by 0 we treat the case where $p=1$ separately. By Definition 8.11 we have 
\begin{align*}
   \int_1^\infty\frac{1}{x}dx = \lim_{y \to \infty} \int_1^y \frac{1}{x}dx=\lim_{y \to \infty}(\ln(y)-\ln(1)) \to \infty .
\end{align*}
In general we have the following: 
\begin{align*}
    \int_1^\infty x^{-p}dx= \lim_{y \to \infty}\int_1^y x^{-p}dx =\lim_{y\to\infty}\Bigg[\frac{x^{-p+1}}{-p+1}\Bigg]_1^y \\
    = \lim_{y \to \infty} \Big(\frac{1}{1-p}\Big)\Big[y^{1-p}-1\Big].
\end{align*}

We see from this that this integral converges for values p where $1-p<0 \implies p>1$ .
\end{answer}


\newpage
\indent \newline
\begin{exer}
Show that $\int_0^\infty x^n e^{-x} dx = n!.$
\end{exer}




\begin{proof}

By induction, Where $n=1$ we have 
\begin{align*}
    \int_0^\infty x^1 e^{-x}dx=\int_0^\infty x(-e^{-x})'dx = [-xe^{-x}]_0^\infty + \int_0^\infty e^{-x}dx\\
    = [0-0] + [-e^{-x}]_0^\infty= -[0 - \frac{1}{e^0}] =1 =1!.
\end{align*}
  Now assuming the condition holds for $n$ we show that it holds for $n+1$:
\begin{align*}
    \int_o^\infty x^{n+1} e^{-x}dx = \int_0^{\infty} x^{n+1}(-e^{-x})'dx = [-x^{n+1}e^{-x}]_0^{\infty} + (n+1)\int_0^\infty x^n e^{-x}dx\\
    = [0 - 0] + (n+1)n! = (n+1)!.
\end{align*}

\end{proof}

\newpage
\indent\newline


\begin{exer}

If $f$ is Riemann integrable on $[a,b]$ show that for every $\epsilon>0$ there are two step functions where 
\begin{align*}
    L(x)\leq f(x)\leq U(x)
\end{align*}
such that
\begin{align*}
    \int_a^b(U(x)-L(x))dx<\epsilon.
\end{align*}

\end{exer}

\begin{answer}

Because $f$ is Riemann integrable on $[a,b]$ we have a partition $[x_0,x_1],...,[x_{n-1},x_n]$ of $[a,b]$ such that for every $\epsilon > 0$ 

\begin{align*}
    \sum_{k=1}^n \omega f([x_{k-1},x_k])(x_k-x_{k-1})< \epsilon.
\end{align*}

Now for that partition we take $\{a_0,...,a_n\}$ to be the points where  $a_k\in[x_{k-1},x_k]$ and $f(a_k)=\inf_{x\in[x_{k-1},x_k]} f(x)$ and the points $\{b_0,...,b_n\}$ to be the points where $b_k\in[x_{k-1},x_k]$ and $f(b_k)=\sup_{x\in[x_{k-1},x_k]} f(x)$. This provides the following step functions:
\begin{equation*}
    L(x) = \begin{cases}
               f(a_1)               & x\in[x_0,x_1]\\
               f(a_2)               & x\in[x_1,x_2]\\
               .              & .\\
               .               & .\\
               .               & .\\
               f(a_n) & x\in[x_{n-1},x_n]
           \end{cases}
\end{equation*}
\begin{equation*}
    U(x) = \begin{cases}
               f(b_1)               & x\in[x_0,x_1]\\
               f(b_2)               & x\in[x_1,x_2]\\
               .              & .\\
               .               & .\\
               .               & .\\
               f(b_n) & x\in[x_{n-1},x_n]
           \end{cases}
\end{equation*}
\indent\newline
where  $L(x)\leq f(x)\leq U(x)$. Moreover, 

\begin{align*}
    \sum_{k=1}^n \omega f([x_{k-1},x_k])(x_k-x_{k-1}) 
    = \Bigg|\sum_{k=1}^n\Big[\sup_{x\in[x_{k-1},x_k]} f(x) - \inf_{x\in[x_{k-1},x_k]} f(x)\Big](x_k-x_{k-1})\Bigg|\\
    =\Bigg|\sum_{k=1}^n U(x)-L(x)(x_k-x_{k-1})\Bigg| =\Bigg|\sum_{k=1}^n U(x)(x_k-x_{k-1})-\sum_{k=1}^n L(x)(x_k-x_{k-1})\Bigg| < \epsilon.
\end{align*}
\newline
Because step functions are integrable(def 8.18) the above inequality meets the Cauchy criteria and that we can write  $\sum_{k=1}^n L(x)(x_k-x_{k-1}) = \int_a^b L(x)dx$ and similarly $\sum_{k=1}^n U(x)(x_k-x_{k-1}) = \int_a^b U(x)dx$. Keeping in mind that we can drop the absolute value because $L(x)\leq f(x)\leq U(x)$ we get the following desired result:
\begin{align*}
    \sum_{k=1}^n U(x)(x_k-x_{k-1})-\sum_{k=1}^n L(x)(x_k-x_{k-1}) = \int_a^b U(x)dx -\int_a^b L(x)dx \\= \int_a^b (U(x)-L(x))dx < \epsilon.
\end{align*}
\end{answer}


\end{document}