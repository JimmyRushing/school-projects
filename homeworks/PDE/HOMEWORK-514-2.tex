\documentclass[12pt,oneside]{amsart}

\title{Math 414/514 Homework 2}
\author{James Rushing}
\date{01/17/20}

% Homework template by Zach Teitler
% Public domain, free to use and modify for any purpose
% v1.0 2020-01-13

% Packages

\usepackage[T1]{fontenc}
\usepackage{amsmath,amsfonts,amssymb,amsthm}
\usepackage[letterpaper,margin=1.5in]{geometry}
\usepackage[pagebackref]{hyperref}
\usepackage{booktabs}
\usepackage{enumitem}

\usepackage{fancyhdr}
\pagestyle{fancy}



% Extra space between lines

\linespread{1.2}



% Theorems, lemmas, etc.


\newtheorem{theorem}[equation]{Theorem}
\newtheorem{claim}[equation]{Claim}
\newtheorem{lemma}[equation]{Theorem}
\newtheorem{corollary}[equation]{Theorem}
\newtheorem{conjecture}[equation]{Conjecture}
\newtheorem{question}[equation]{Question}

\theoremstyle{definition}

\newtheorem{definition}[equation]{Definition}

\theoremstyle{remark}

\newtheorem{exer}{Exercise}
\newtheorem{remark}[equation]{Remark}
\newtheorem{example}[equation]{Example}

\numberwithin{equation}{exer}


% Answers

% Use "proof" for proof exercises!
% Use "answer" for question (non-proof) exercises

\newenvironment{answer}{\bigskip\noindent\emph{Answer.}}{\hfill$\diamond$\newline}


% Symbols

\newcommand{\bbC}{\mathbb{C}}
\newcommand{\bbN}{\mathbb{N}}
\newcommand{\bbR}{\mathbb{R}}
\newcommand{\bbQ}{\mathbb{Q}}
\newcommand{\bbZ}{\mathbb{Z}}




\begin{document}
\maketitle

\begin{exer} (8.3.2)

If $f$ is continuous, show that there is a point $\xi$ in (a,b) s.t. 
\begin{align*}
    \int_a^bf(x)dx = f(\xi)(a-b).
\end{align*}

\end{exer}



\




\begin{answer}

By 8.8 we have a function $F(x)$ s.t. $F'(x)=f(x)$ for $x\in[a,b]$. By the MVT we have a value $\xi$ s.t.
\begin{align*}
    F'(\xi)=\frac{F(b)-F(a)}{b-a}\\
    \implies F'(\xi)(b-a)=F(b)-F(a).
\end{align*}
Then, using 8.9 gives us
\begin{align*}
    F'(\xi)(b-a)=F(b)-F(a)=\int_a^bF'(x)dx=\int_a^bf(x)dx.
\end{align*}

\end{answer}


\newpage



\begin{exer} (8.3.4)

Show that if $f$ is continuous and non-negative on $[a,b]$ and 
\begin{align*}
    \int_a^bf(x)dx=0
\end{align*}
show that $f(x)=0$.
\end{exer}


\begin{proof}

By 8.8 we have a function $F(x)=\int_a^xf(t)dt$ s.t. that $F'(x)=f(x)$. Because $f(x)$ is non-negative, we observe that F(x) is an increasing function. However, because $\int_a^bf(x)dx=0$, we know that F(a) and F(b) both equal 0. Therefore $F(x)=0$ for all $x\in [a,b]$ or $f(x)=0$ for all $x\in [a,b]$.



  

\end{proof}

\newpage



\begin{exer} (8.3.6)

If $f$ is continuous on $[a,b]$ and $\int_a^bf(x)g(x)dx=0$ for every continuous  $g$ on $[a,b]$, then $f(x)=0$ for $x\in[a,b]$

\end{exer}

\begin{proof}
If the conclusion holds for all continuous functions $g(x)$ then it holds where $g(x) = f(x)$. This gives us
\begin{align*}
    \int_a^b(f(x))^2dx = 0
\end{align*}
Using 8.3.4, where $p(x) = (f(x))^2$ is non-negative we have
\begin{align*}
    \int_a^bp(x)dx = 0 \implies p(x)=(f(x))^2=0\\
    \implies f(x) =0\indent\forall x \in [a,b]
\end{align*}
\end{proof}


\end{document}
