\documentclass[12pt,oneside]{amsart}

\title{Math 414/514 Short Paper}
\author{James Rushing}
\date{02/07/20}

% Homework template by Zach Teitler
% Public domain, free to use and modify for any purpose
% v1.0 2020-01-13

% Packages

\usepackage[T1]{fontenc}
\usepackage{amsmath,amsfonts,amssymb,amsthm}
\usepackage[letterpaper,margin=1.5in]{geometry}
\usepackage[pagebackref]{hyperref}
\usepackage{booktabs}
\usepackage{enumitem}
\usepackage{fancyhdr}
\usepackage{amsrefs}
\pagestyle{fancy}



% Extra space between lines

\linespread{1.2}



% Theorems, lemmas, etc.


\newtheorem{theorem}[equation]{Theorem}
\newtheorem{claim}[equation]{Claim}
\newtheorem{lemma}[equation]{Theorem}
\newtheorem{corollary}[equation]{Theorem}
\newtheorem{conjecture}[equation]{Conjecture}
\newtheorem{question}[equation]{Question}

\theoremstyle{definition}

\newtheorem{definition}[equation]{Definition}

\theoremstyle{remark}

\newtheorem{exer}{Exercise}
\newtheorem{remark}[equation]{Remark}
\newtheorem{example}[equation]{Example}

\numberwithin{equation}{exer}


% Answers

% Use "proof" for proof exercises!
% Use "answer" for question (non-proof) exercises

\newenvironment{answer}{\bigskip\noindent\emph{Answer.}}{\hfill$\diamond$\newline}


% Symbols

\newcommand{\bbC}{\mathbb{C}}
\newcommand{\bbN}{\mathbb{N}}
\newcommand{\bbR}{\mathbb{R}}
\newcommand{\bbQ}{\mathbb{Q}}
\newcommand{\bbZ}{\mathbb{Z}}




\begin{document}
\maketitle
\indent Understanding how the first transcendental number came to be can give insight into an ongoing mathematical scavenger hunt for ever more irrational numbers. The story of the number $e$ begins in earnest in the early 17th century. Although it wasn't to be named explicitly until the following century, its properties were being explored peripherally through three broader mathematical developments: the invention of logarithms, the invention of differential calculus, and advancements in the understanding of series. These advancements coalesced in the 18th century, largely due to the efforts of Leonhard Euler, into an understanding of $e$ as an actual number. Finally, in the late 18th century, the representation of $e$ as in infinite series provided the tools to investigate the number $e$ directly and provided a concise proof that $e$ was a real, irrational number \cite{numbere}. This opened the door to the pursuit of ever more irrational (and subsequently transcendental) numbers .  
\newline
\indent The mathematician accredited with the aforementioned proof was J.B Fourier. He employed a proof by contradiction. We will attempt a version of his proof here. We take for granted what Euler had all ready shown, which was that the number $e$ can be expressed as follows:
\begin{align*}
    e =  1 +\frac{1}{1!} +\frac{1}{2!} + \frac{1}{3!}+ \ldots
\end{align*}
First, we notice from the first two terms of the sum that $e$ must be larger than $2$. Next we compare this series to a series in which we have strategically replaced denominators with smaller numbers we see that
\begin{align*}
    e = 1 +\frac{1}{1!} +\frac{1}{2!} + \frac{1}{3!}+ \ldots < 1 + 1 +\frac{1}{2} +\frac{1}{2^2} + \frac{1}{2^3}+ \ldots = 1 + \sum_{n=0}^{\infty}\frac{1}{2^n} =3.
\end{align*}
Because $e$ is strictly between $2$ and $3$, it is not an integer. Let us assume that $e$ is a rational number, and lead ourselves to a contradiction. If $e$ is rational, we have numbers $a,b\in\mathbb{Z}$ such that $e=\frac{a}{b}$. Furthermore, the rationality of $e$ tells us that $eb!=\frac{ab!}{b}=a(b-1)!$ must be an integer. With that, we write our new found integer in accordance with our original function as follows:
\begin{align*}
    eb! = b!\Big[1 + 1 +\frac{1}{1!} +\frac{1}{2!} + \frac{1}{3!}+ \ldots \Big].
\end{align*}
\indent Now, for reasons that will become clear later, we break this sum into a two sums and distribute $b!$ as follows:
\begin{align*}
    eb!=b!\Big[1+1+ \frac{1}{2!}+\ldots+\frac{1}{b!}\Big]+b!\Big[\frac{1}{(b+1)!}+\frac{1}{(b+2)!}+\ldots \Big]
\end{align*}
Distributing $b!$ into the left sum gives a a sum of integers which is itself an integer. Therefore, we have $eb!=N +b!\Big[\frac{1}{(b+1)!}+\frac{1}{(b+2)!}+\ldots \Big] $ where $N$ and $eb!$ are both integers. \newline
\indent If we can show that the right sum is between $0$ and $1$, and therefore not an integer, then we would have a glaring contradiction. It is greater than $0$ by virtue of being a sum of all positive terms. After distributing $b!$ we again strategically replace denominators with smaller numbers to obtain 
\begin{align*}
    \Big[\frac{1}{b+1}+\frac{1}{(b+1)(b+2)}+ \frac{1}{(b+1)(b+2)(b+3)}+\ldots\Big]\\ < \Big[\frac{1}{b+1}+\frac{1}{(b+1)(b+1)}+ \frac{1}{(b+1)(b+1)(b+1)}+\ldots\Big]\\ = \sum_{n=1}^{\infty}\frac{1}{(b+1)^n} = \frac{1}{b} \leq 1.
\end{align*}
Therefore, our assumption that $e$ was a rational number was false. 

\indent 
Fourier's version of this proof was published in 1815. By this time, it could be shown to anyone, with a firm grasp of addition and multiplication, that the number $e$ cannot be written as a ratio of integers. Stating this another way motivates deeper investigations of the number $e$ throughout history. If the number $e$ cannot be written as a ratio of integers $\Big(e=\frac{a}{b}\Big)$ then it cannot be written as the solution of a linear equation with non-zero coefficients $\Big(a- bx = 0\Big)$. Can $e$ be written as the solution of a non-zero quadratic equation? A non-zero polynomial?
\newline
\indent 
This final question ushers us into modernity. Using similar methods to the proof outlined above, Joseph Liouville proved that $e$ cannot be written as a solution to a non-zero quadratic equation. Our question about polynomials proved more difficult. It wasn't until 1873 that Charles Hermite proved that the number $e$ cannot be written as the solution of an integer polynomial. Numbers with this property are called transcendental numbers, and the number $e$ was the first number proven to be transcendental. Five years later, F. Lindemann proved that $\pi$ was transcendental.
\newline
\indent
The project of conceptualizing and proving these attributes of the number $e$ took the best mathematical minds three generations to complete. In fact, this type of work continues today, albeit for other numbers. For example, the story of the numbers produced by the \textit{Riemann zeta function},
\begin{align*}
    \zeta (s) = \sum_{k=1}^{\infty} \frac{1}{k^s}= 1 + \frac{1}{2^s}+\frac{1}{3^s}+...,\indent s=2,3,4,...
\end{align*}
began around the same time as that of the number $e$. Euler showed solutions for the following special cases where $s=2,4$:
\begin{align*}
    \zeta(2)=\frac{\pi^2}{6} \\
    \zeta(4) = \frac{\pi^4}{90}.
\end{align*}
In general, Euler knew the \textit{Riemann zeta function} values for even inputs were $\zeta (2n)=a_n \pi^{2n}$, where $a_n$ is rational. As a result, the \textit{Riemann zeta function} for even inputs were known to be irrational. The \textit{Riemann zeta function} with odd inputs remained completely mysterious until 1978, when seemingly out of the blue, a French mathematician named Roger Ap\'ery proved the irrationality of $\zeta (3)$. It has yet to proven whether or not $\zeta(3)$ is transcendental, and furthermore, irrationality has yet to be proven for any \textit{Riemann zeta function} value of the form $\zeta(2n-1)$ \cite{ross}.
\newline
\indent
The bombshell proof of the irrationality of $\zeta(3)$ is just one of the exciting proofs in this field in the last 50 years. The following numbers were all proven to be transcendental in the last 50 years:
\begin{align*}
    \sin{1}\indent\indent\indent \text{Hardy and Wright (1979)}\\
    \ln{2} \indent\indent\indent \text{Hardy and Wright (1979)}\\
    2^{\sqrt{2}}\indent\indent\indent \text{Hardy and Wright (1979)}\\
    e^{\pi \sqrt{d}} ,d\in\mathbb{Z}^+ \indent\indent \indent \text{Nesterenko (1999)}.
\end{align*}
 $\pi^{\pi}$,$e^e$,$\pi^e$ are a few of the interesting numbers that are yet to be proven to be transcendental \cite{transcendental}.
\newpage
\begin{bibdiv}
\begin{biblist}
\bib{transcendental}{webpage}{
  author={Weisstein, Eric W.},
  title={Transcendental Number},
  note={From \textit{MathWorld}--A Wolfram Web Resource.}
  url={http://mathworld.wolfram.com/TranscendentalNumber.html},
  accessdate={February 07, 2020},
}

\bib{ross}{article}{
  author={Ross, Marty},
  title={Irrational Thoughts},
  date={March, 2004},
  journal={The Mathematical Gazette},
  volume={88},
  number={551},
  pages={68--78},
}
\bib{numbere}{article}{
  author={Strain, M.},
  author={Mitchell, U.G.},
  title={The Number $e$},
  date={January, 1936},
  journal={Osiris},
  volume={1},
  pages={476--496},
}
\bib{numbere}{article}{ Title = {ovito},
Author = {Stukowski, Alexander},
Title = {{Visualization and analysis of atomistic simulation data with OVITO-the
   Open Visualization Tool}},
Journal = {{MODELLING AND SIMULATION IN MATERIALS SCIENCE AND ENGINEERING}},
Year = {{2010}},
Volume = {{18}},
Number = {{1}},
Month = {{JAN}},
DOI = {{10.1088/0965-0393/18/1/015012}},
Article-Number = {{015012}},
ISSN = {{0965-0393}},
EISSN = {{1361-651X}},
ResearcherID-Numbers = {{Stukowski, Alexander/G-9695-2017}},
ORCID-Numbers = {{Stukowski, Alexander/0000-0001-6750-3401}},
Unique-ID = {{ISI:000272791800012}},
}

}

\end{biblist}
\end{bibdiv}
\end{document}

