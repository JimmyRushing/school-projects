\documentclass[12pt,oneside]{amsart}

\title{Math 414/514 Homework 5}
\author{James Rushing}
\date{02/22/20}

% Homework template by Zach Teitler
% Public domain, free to use and modify for any purpose
% v1.0 2020-01-13

% Packages

\usepackage[T1]{fontenc}
\usepackage{amsmath,amsfonts,amssymb,amsthm}
\usepackage[letterpaper,margin=1.5in]{geometry}
\usepackage[pagebackref]{hyperref}
\usepackage{booktabs}
\usepackage{enumitem}

\usepackage{fancyhdr}
\pagestyle{fancy}



% Extra space between lines

\linespread{1.2}



% Theorems, lemmas, etc.


\newtheorem{theorem}[equation]{Theorem}
\newtheorem{claim}[equation]{Claim}
\newtheorem{lemma}[equation]{Theorem}
\newtheorem{corollary}[equation]{Theorem}
\newtheorem{conjecture}[equation]{Conjecture}
\newtheorem{question}[equation]{Question}

\theoremstyle{definition}

\newtheorem{definition}[equation]{Definition}

\theoremstyle{remark}

\newtheorem{exer}{Exercise}
\newtheorem{remark}[equation]{Remark}
\newtheorem{example}[equation]{Example}

\numberwithin{equation}{exer}


% Answers

% Use "proof" for proof exercises!
% Use "answer" for question (non-proof) exercises

\newenvironment{answer}{\bigskip\noindent\emph{Answer.}}{\hfill$\diamond$\newline}


% Symbols

\newcommand{\bbC}{\mathbb{C}}
\newcommand{\bbN}{\mathbb{N}}
\newcommand{\bbR}{\mathbb{R}}
\newcommand{\bbQ}{\mathbb{Q}}
\newcommand{\bbZ}{\mathbb{Z}}




\begin{document}
\maketitle

\begin{exer} (9.2.5)
Verify that $9.4,9.5,9.6$ can be paraphrased as "the order of taking the limit matters."
\end{exer}



\




\begin{answer}

\indent In examples 9.4 and 9.5 we see that the order of taking the limit matters, in that, if we first take the limit of $f_n(x)=\frac{x^n}{n}$ we get 0. Differentiating this gives 0. But if we first take the derivative and then take the limit we end up with example 9.4 which is not 0 for $x \in [0,1]$
\newline \indent In example 9.6 we see that order matters,in that, integrating $f_n(x)$ after taking the limit isn't necessarily the same as taking the limit first and then integrating.

\end{answer}


\newpage
\indent \newline
\begin{exer} (9.2.11)
Show that the set of convergence points of $\{f_n\}$ can be written as 
\begin{align*}
    E= \bigcap_{k=1}^\infty \bigcup_{N=1}^\infty \bigcap_{n=N}^\infty \bigcap_{m=N}^\infty \{ x : |f_n(x)-f_m(x)|\leq \frac{1}{k}\}.
    \end{align*}
\end{exer}
\begin{answer}
    To show this we observe first that if $\{f_n\}$ is convergent on $E$ then it is Cauchy on E. Or $\forall$ $\epsilon > 0$ $\exists$ and $N$ such that where $n,m \geq N$ we have 
\begin{align*}
    |f(_n(x) - f_m(x)| < \epsilon.
\end{align*}
If this is true for $\epsilon > 0$ then it it is true for $\epsilon = \frac{1}{k}$ $\forall$ $k \in \mathbb{N}$. So we can write the points where $f_n(x)$ converges as
\begin{align*}
    E 
    =\bigcap_{k=1}^\infty \{x: \exists N, \forall n,m\geq N,|f(_n(x) - f_m(x)| < \frac{1}{k} \}\\
    =\bigcap_{k=1}^\infty \bigcup_{N=1}^\infty \bigcap_{n,m\geq N}^\infty\{x: |f(_n(x) - f_m(x)| < \frac{1}{k} \}\\
    =\bigcap_{k=1}^\infty \bigcup_{N=1}^\infty \bigcap_{n=N}^\infty \bigcap_{m=N}^\infty\{x: |f(_n(x) - f_m(x)| < \frac{1}{k} \}.
\end{align*}

\end{answer}



\newpage

\indent \newline

\begin{exer} (9.3.4)
Prove that if $f_n \to f$ uniformly on $E_1$ and $E_2$, then $f_n \to f$ on $E_1 \cup E_2$.

\end{exer}

\begin{proof}
If $f_n \to f$ uniformly on $E_1$ and $E_2$ then we have $N_1,N_2 \in \mathbb{N}$ s.t. 
\begin{align*}
    |f_n(x) - f(x)| < \epsilon \\
    \forall x \in E_1 \indent \text{and} \indent n\geq N_1
    \end{align*}
\begin{align*}
    \textbf{and}
\end{align*}
\begin{align*}
    |f_n(x) - f(x)| < \epsilon \\
    \forall x \in E_2 \indent \text{and} \indent n\geq N_2.
\end{align*}
Choose $N$ to be the greater of $N_1$ and $N_2$. Now we have
\begin{align*}
    |f_n(x) - f(x)| < \epsilon\\
        \forall x \in E_1\cup E_2 \indent \text{and} \indent n\geq N.
\end{align*}
\end{proof}
\newpage
\indent \newline
\begin{exer} (9.3.5)
Prove or disprove that if $f_n \to f$ uniformly on each $E_1,E_2,,E_3...,$ then $f_n \to f$ uniformly on $\cap_{k=1}^\infty E_k$.
\end{exer}
\begin{answer}
\indent As a counterexample, consider $f_n = x^n$. Let $E_1 = \{1\}$ and let $E_k = [0, 1-\frac{1}{k}]$ for $k= 2,3,4,..$. Now we see that $f_n = x^n$ converges uniformly on each $E_k$. However 
\begin{align*}
    \cap_{k=1}^\infty E_k = [0,1]
\end{align*}
and from example $9.4$ we know that $f_n$ does not uniformly converge on $[0,1]$.
\end{answer}
\end{document}
